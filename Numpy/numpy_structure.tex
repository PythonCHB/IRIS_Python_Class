\documentclass{beamer}
%\usepackage[latin1]{inputenc}
\usetheme{Warsaw}
\title[Intro to Python]{Numpy}
\author{Christopher Barker}
\institute{IRIS}
\date{October 23, 2013}

\usepackage{listings}
\usepackage{hyperref}

\begin{document}

% ---------------------------------------------
\begin{frame}
  \titlepage
\end{frame}

% ---------------------------------------------
\begin{frame}
\frametitle{Table of Contents}
%\tableofcontents[currentsection]
  \tableofcontents
\end{frame}

%%%%%%%%%%%%%%%%%%%%%%
\section{numpy}

%---------------------------
\begin{frame}[fragile]{numpy}

\vfill
\begin{center}
{\LARGE numpy}

\vfill
{\Large Not just for lots of numbers!\\[0.1in]
(but it's great for that!)}
\end{center}
\vfill
\url{http://www.numpy.org/}
\end{frame} 


\begin{frame}[fragile]{what is numpy?}

{\Large An N-Dimensional array object}

\vfill
{\Large A whole pile of tools for operations on/with that object.}

\vfill

\end{frame} 

%----------------------------------
\begin{frame}[fragile]{Why numpy?}

{\Large Classic answer: Lots of numbers}

\vfill
\begin{itemize}
  \item Faster
  \item Less memory
  \item More data types
\end{itemize}


\vfill
{\Large Even if you don't have lot of numbers:}
\begin{itemize}
  \item N-d array slicing
  \item Vector operations
  \item Flexible data types
\end{itemize}

\end{frame} 

% %----------------------------------
\begin{frame}[fragile]{why numpy?}

{\Large Wrapper for a block of memory:}

\begin{itemize}
  \item Interfacing with C libs
  \item PyOpenGL
  \item GDAL
  \item NetCDF4
  \item Shapely
\end{itemize}

{\Large Image processing:}
\begin{itemize}
  \item PIL
  \item WxImage
  \item ndimage
\end{itemize}

\end{frame} 

%----------------------------------
\begin{frame}[fragile]{What is an nd array?}
\begin{itemize}
  \item N-dimensional (up to 32!)
  \item Homogeneous array:
  \begin{itemize}
    \item Every element is the same type\\
          (but that type can be a pyObject)
    \item Int, float, char -- more exotic types
  \end{itemize}
  \item ``rank'' – number of dimensions
  \item Strided data:
  \begin{itemize}
    \item Describes how to index into block of memory
    \item PEP 3118 -- Revising the buffer protocol
  \end{itemize}
\end{itemize}

\vfill
{\large demos: \verb|memory.py| and \verb|structure.py|}

\end{frame} 

%----------------------------------
\begin{frame}[fragile]{Built-in Data Types}

\begin{itemize}
  \item Signed and unsigned Integers\\
        8, 16, 32, 64 bits
  \item Floating Point\\
        32, 64, 96, 128 bits (not all platforms)
  \item Complex\\
        64, 128, 192, 256 bits
  \item String and unicode\\
        Static length 
  \item Bool \\
        8 bit
  \item Python Object \\
        Really a pointer
\end{itemize}

\vfill
{\large demo: \verb|object.py|}

\end{frame} 

%----------------------------------
\begin{frame}[fragile]{Compund dtypes}

{\Large
\begin{itemize}
  \item Can define any combination of other types \\
        Still Homogenous:  Array of structs.
  \item Can name the fields
  \item Can be like a database table
  \item Useful for reading binary data
\end{itemize}
}

\vfill
{\Large demo: \verb|dtypes.py|}

\end{frame} 

%----------------------------------
\begin{frame}[fragile]{Array Constructors:}

{\Large From scratch:}

\verb|ones(), zeros(), empty(), arange(), linspace(), logspace()|

( Default dtype: np.float64 )


\vfill
{\Large From sequences:}

\verb|array(), asarray()|

( Build from any sequence )

\vfill
{\Large  From binary data:}
\verb|fromstring(), frombuffer(), fromfile()|

\vfill
{\Large Assorted linear algebra standards:}
\verb|eye(), diag(), etc.| 

\vfill
{\large demo: \verb|constructors.py|}
\end{frame} 

%----------------------------------
\begin{frame}[fragile]{Broadcasting:}

{\Large Element-wise operations among two different rank arrays:}

\vfill
{\Large Simple case: scalar and array:}

\begin{verbatim}
In [37]: a
Out[37]: array([1, 2, 3])

In [38]: a*3
Out[38]: array([3, 6, 9])
\end{verbatim}

{\Large Great for functions of more than one variable on a grid}

\vfill
{\large demo: \verb|broadcasting.py|}
\end{frame} 

%----------------------------------
\begin{frame}[fragile]{Slicing -- views:}

{\Large a slice is a ``view'' on the array:\\
new object, but shares memory:}

\begin{verbatim}
In [12]: a = np.array((1,2,3,4))
In [13]: b = a[:]
# for lists -- [:] means copy -- not for arrays!
In [15]: a is b
Out[15]: False
# it's new array, but...
In [16]: b[2] = 5
In [17]: a
Out[17]: array([1, 2, 5, 4])
# a and b share data
\end{verbatim}

\vfill
{\large demo: \verb|slice.py|}
\end{frame} 


%----------------------------------
\begin{frame}[fragile]{Working with compiled code:}

{\Large Wrapper around a C pointer to a block of data}

\begin{itemize}
  \item Some code can't be vectorized
  \item Interface with existing libraries
\end{itemize}

\vfill
{\Large Tools:}
\begin{itemize}
  \item C API: you don't want to do that!
  \item Cython: typed arrays
  \item Ctypes 
  \item SWIG: numpy.i 
  \item Boost: boost array
  \item f2py
\end{itemize}

\vfill
Example of numpy+cython: \url{http://wiki.cython.org/examples/mandelbrot}
\end{frame} 

%----------------------------------
\begin{frame}[fragile]{numpy persistance:}

{\Large \verb|.tofile() / fromfile()|}\\
 -- Just the raw bytes, no metadata

\vfill
{\Large pickle }

\vfill
{\Large \verb|savez()| -- numpy zip format}\\
Compact: binary dump plus metadata

\vfill
{\Large netcdf}

\vfill
{\Large Hdf}
\begin{itemize}
  \item Pyhdf
  \item pytables
\end{itemize}


\end{frame} 

%----------------------------------
\begin{frame}[fragile]{Other stuff:}

\begin{itemize}
  \item Masked arrays
  \item Memory-mapped files
  \item Set operations: unique, etc
  \item Random numbers
  \item Polynomials
  \item FFT
  \item Sorting and searching
  \item Linear Algebra
  \item Statistics
\end{itemize}

{\large (And all of scipy!)}

\end{frame} 

%----------------------------------
\begin{frame}[fragile]{numpy docs:}

\url{www.numpy.org}\\
   -- Numpy reference Downloads, etc

\vfill
\url{www.scipy.org}\\
   -- lots of docs

\vfill
{\large Scipy cookbook}\\
\url{http://www.scipy.org/Cookbook}

\vfill
{\large ``The Numpy Book''}\\
\url{http://www.tramy.us/numpybook.pdf}

\end{frame} 

\end{document}
 